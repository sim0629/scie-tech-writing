\documentclass[11pt,a4paper]{article}

\usepackage{kotex}
\usepackage{datetime}
\usepackage{framed}
\usepackage{fullpage}

\newdateformat{koreandate}{\THEYEAR년 \twodigit{\THEMONTH}월 \twodigit{\THEDAY}일}

\begin{document}

\title{예약한 강의실의 접근성 및 이용 편의성 개선 방안}
\author{
김슬기\\
김재찬\\
김찬민\\
심규민\\
유재성
}
\date{}
\maketitle

\begin{framed}

\centerline{\textless배움의 윤리 서약\textgreater}

\begin{enumerate}
\item 이 과제물은 내가(우리가) 직접 연구하여 작성한 것이다.
\item 정확한 출처 제시 없이 다른 사람의 글이나 생각을 가져오지 않았다.
\item 인용한 문헌의 내용이나 자료(도표나 데이터)를 조작(위조 혹은 변조)하지 않았다.
\item 과제물을 다른 사람으로부터 받거나 구매하여 제출하지 않았다.
\item 과제물 작성에 참여하지 않은 사람을 공동 제출자로 명기하지 않았다.
\end{enumerate}

\begin{center}
이 과제물은 위의 항목들을 준수하여 작성한 것임을 확인합니다.\\
\hfill\break
\koreandate\today\\
\hfill\break
작성자\\
\hfill\break
김슬기 (서명)\\
김재찬 (서명)\\
김찬민 (서명)\\
심규민 (서명)\\
유재성 (서명)
\end{center}

\end{framed}

\renewcommand{\abstractname}{초록}
\begin{abstract}
본 논문에서는 서울대학교 강의실 예약 시스템을 통해 예약한 강의실의 접근성과 그 강의실을 이용할 때까지 발생할 수 있는 불편함을 해소하여 편의성을 개선할 방안을 제시한다. 기존의 시스템은 제삼자인 수위의 도움을 받아야 문을 열 수 있는 등 강의실 접근성이 떨어지며, 인증을 해주는 수위가 자리를 비울 경우 강의실을 이용하지 못할 수도 있다. 본 연구에서는 우선 설문조사를 통해 이러한 문제가 실제로 존재하는지 보였다. 예약한 강의실 문이 잠겨 있는 것을 경험한 사용자는 전체의 절반이나 되었다. 문제를 해결할 수 있는 방안으로 블랙 리스트 모델과 화이트 리스트 모델을 제시하였다. 제시한 모델이 문제를 잘 해결할 수 있는지 검증하기 위해, 기존 모델과 제시한 모델들을 형식화하고 이를 프롤로그 언어로 프로그램을 작성하여 비교하였다. 그 결과 약한 블랙 리스트 모델은 오히려 기존 모델보다 불편하였고, 강한 블랙 리스트 모델과 화이트 리스트 모델은 모두 기존 모델의 기능을 잘 수행하면서, 불편한 점을 개선할 수 있었다.\\
\centerline{핵심어: 강의실, 수위, 예약, 잠금장치, 일회성 비밀번호, 이미지 프로세싱}
\end{abstract}

\end{document}
